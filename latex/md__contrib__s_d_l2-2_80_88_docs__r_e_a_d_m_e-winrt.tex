This port allows S\+DL applications to run on Microsoft\textquotesingle{}s platforms that require use of \char`\"{}\+Windows Runtime\char`\"{}, aka. \char`\"{}\+Win\+R\+T\char`\"{}, A\+P\+Is. Microsoft may, in some cases, refer to them as either \char`\"{}\+Windows Store\char`\"{}, or for Windows 10, \char`\"{}\+U\+W\+P\char`\"{} apps.

Some of the operating systems that include Win\+RT, are\+:


\begin{DoxyItemize}
\item Windows 10, via its Universal Windows Platform (U\+WP) A\+P\+Is
\item Windows 8.\+x
\item Windows RT 8.\+x (aka. Windows 8.\+x for A\+RM processors)
\item Windows Phone 8.\+x
\end{DoxyItemize}

\subsection*{Requirements }


\begin{DoxyItemize}
\item Microsoft Visual C++ (aka Visual Studio), either 2017, 2015, 2013, or 2012
\begin{DoxyItemize}
\item Free, \char`\"{}\+Community\char`\"{} or \char`\"{}\+Express\char`\"{} editions may be used, so long as they include support for either \char`\"{}\+Windows Store\char`\"{} or \char`\"{}\+Windows Phone\char`\"{} apps. \char`\"{}\+Express\char`\"{} versions marked as supporting \char`\"{}\+Windows Desktop\char`\"{} development typically do not include support for creating Win\+RT apps, to note. (The \char`\"{}\+Community\char`\"{} editions of Visual C++ do, however, support both desktop/\+Win32 and Win\+RT development).
\item Visual Studio 2017 can be used, however it is recommended that you install the Visual C++ 2015 build tools. These build tools can be installed using VS 2017\textquotesingle{}s installer. Be sure to also install the workload for \char`\"{}\+Universal Windows Platform development\char`\"{}, its optional component, the \char`\"{}\+C++ Universal Windows Platform tools\char`\"{}, and for U\+WP / Windows 10 development, the \char`\"{}\+Windows 10 S\+D\+K (10.\+0.\+10240.\+0)\char`\"{}. Please note that targeting U\+WP / Windows 10 apps from development machine(s) running earlier versions of Windows, such as Windows 7, is not always supported by Visual Studio, and you may get error(s) when attempting to do so.
\item Visual C++ 2012 can only build apps that target versions 8.\+0 of Windows, or Windows Phone. 8.\+0-\/targeted apps will run on devices running 8.\+1 editions of Windows, however they will not be able to take advantage of 8.\+1-\/specific features.
\item Visual C++ 2013 cannot create app projects that target Windows 8.\+0. Visual C++ 2013 Update 4, can create app projects for Windows Phone 8.\+0, Windows Phone 8.\+1, and Windows 8.\+1, but not Windows 8.\+0. An optional Visual Studio add-\/in, \char`\"{}\+Tools for Maintaining Store apps for Windows 8\char`\"{}, allows Visual C++ 2013 to load and build Windows 8.\+0 projects that were created with Visual C++ 2012, so long as Visual C++ 2012 is installed on the same machine. More details on targeting different versions of Windows can found at the following web pages\+:
\begin{DoxyItemize}
\item \href{http://msdn.microsoft.com/en-us/library/windows/apps/br211384.aspx}{\tt Develop apps by using Visual Studio 2013}
\item \href{http://msdn.microsoft.com/en-us/library/windows/apps/dn263114.aspx#AddMaintenanceTools}{\tt To add the Tools for Maintaining Store apps for Windows 8}
\end{DoxyItemize}
\end{DoxyItemize}
\item A valid Microsoft account -\/ This requirement is not imposed by S\+DL, but rather by Microsoft\textquotesingle{}s Visual C++ toolchain. This is required to launch or debug apps.
\end{DoxyItemize}

\subsection*{Status }

Here is a rough list of what works, and what doesn\textquotesingle{}t\+:


\begin{DoxyItemize}
\item What works\+:
\begin{DoxyItemize}
\item compilation via Visual C++ 2012 through 2015
\item compile-\/time platform detection for S\+DL programs. The C/\+C++ \#define, {\ttfamily \+\_\+\+\_\+\+W\+I\+N\+R\+T\+\_\+\+\_\+}, will be set to 1 (by S\+DL) when compiling for Win\+RT.
\item G\+P\+U-\/accelerated 2D rendering, via S\+D\+L\+\_\+\+Renderer.
\item Open\+GL ES 2, via the A\+N\+G\+LE library (included separately from S\+DL)
\item software rendering, via either \mbox{\hyperlink{struct_s_d_l___surface}{S\+D\+L\+\_\+\+Surface}} (optionally in conjunction with \mbox{\hyperlink{_s_d_l__video_8h_a0f7a1d102e1d5dd2d739ad70fe268385}{S\+D\+L\+\_\+\+Get\+Window\+Surface()}} and \mbox{\hyperlink{_s_d_l__video_8h_a7bbfee05788dc85a1d67218cf3006653}{S\+D\+L\+\_\+\+Update\+Window\+Surface()}}) or via the S\+D\+L\+\_\+\+Renderer A\+P\+Is
\item threads
\item timers (via \mbox{\hyperlink{_s_d_l__timer_8h_a0b9bc71d6287e0ffafdc3419760fe2b3}{S\+D\+L\+\_\+\+Get\+Ticks()}}, \mbox{\hyperlink{_s_d_l__timer_8h_a56ceea49587e3fa5796b2e4bf85603b8}{S\+D\+L\+\_\+\+Add\+Timer()}}, \mbox{\hyperlink{_s_d_l__timer_8h_a2dbeb63c4f0564811a4adf3938808977}{S\+D\+L\+\_\+\+Get\+Performance\+Counter()}}, \mbox{\hyperlink{_s_d_l__timer_8h_a507ebea12e31dacc9f85f7d9febe0efb}{S\+D\+L\+\_\+\+Get\+Performance\+Frequency()}}, etc.)
\item file I/O via \mbox{\hyperlink{struct_s_d_l___r_wops}{S\+D\+L\+\_\+\+R\+Wops}}
\item mouse input (unsupported on Windows Phone)
\item audio, via S\+DL\textquotesingle{}s W\+A\+S\+A\+PI backend (if you want to record, your app must have \char`\"{}\+Microphone\char`\"{} capabilities enabled in its manifest, and the user must not have blocked access. Otherwise, capture devices will fail to work, presenting as a device disconnect shortly after opening it.)
\item .D\+LL file loading. Libraries {\itshape M\+U\+ST} be packaged inside applications. Loading anything outside of the app is not supported.
\item system path retrieval via S\+DL\textquotesingle{}s filesystem A\+P\+Is
\item game controllers. Support is provided via the S\+D\+L\+\_\+\+Joystick and S\+D\+L\+\_\+\+Game\+Controller A\+P\+Is, and is backed by Microsoft\textquotesingle{}s X\+Input A\+PI. Please note, however, that Windows limits game-\/controller support in U\+WP apps to, \char`\"{}\+Xbox compatible controllers\char`\"{} (many controllers that work in Win32 apps, do not work in U\+WP, due to restrictions in U\+WP itself.)
\item multi-\/touch input
\item app events. S\+D\+L\+\_\+\+A\+P\+P\+\_\+\+W\+I\+L\+L\+E\+N\+T\+E\+R$\ast$ and S\+D\+L\+\_\+\+A\+P\+P\+\_\+\+D\+I\+D\+E\+N\+T\+E\+R$\ast$ events get sent out as appropriate.
\item window events
\item using Direct3D 11.\+x A\+P\+Is outside of S\+DL. Non-\/\+X\+A\+ML / Direct3\+D-\/only apps can choose to render content directly via Direct3D, using S\+DL to manage the internal Win\+RT window, as well as input and audio. (Use \mbox{\hyperlink{_s_d_l__syswm_8h_ad6e40121a96c88af20d9469a04706fef}{S\+D\+L\+\_\+\+Get\+Window\+W\+M\+Info()}} to get the Win\+RT \textquotesingle{}Core\+Window\textquotesingle{}, and pass it into I\+D\+X\+G\+I\+Factory2\+::\+Create\+Swap\+Chain\+For\+Core\+Window() as appropriate.)
\end{DoxyItemize}
\item What partially works\+:
\begin{DoxyItemize}
\item keyboard input. Most of Win\+RT\textquotesingle{}s documented virtual keys are supported, as well as many keys with documented hardware scancodes. Converting S\+D\+L\+\_\+\+Scancodes to or from S\+D\+L\+\_\+\+Keycodes may not work, due to missing A\+P\+Is (Map\+Virtual\+Key()) in Microsoft\textquotesingle{}s Windows Store / U\+WP A\+P\+Is.
\item S\+D\+Lmain. Win\+RT uses a different signature for each app\textquotesingle{}s \mbox{\hyperlink{_c_make_c_compiler_id_8c_a0ddf1224851353fc92bfbff6f499fa97}{main()}} function. S\+D\+L-\/based apps that use this port must compile in S\+D\+L\+\_\+winrt\+\_\+main\+\_\+\+Non\+X\+A\+M\+L.\+cpp (in {\ttfamily S\+DL\textbackslash{}src\textbackslash{}main\textbackslash{}winrt\textbackslash{}}) directly in order for their C-\/style \mbox{\hyperlink{_c_make_c_compiler_id_8c_a0ddf1224851353fc92bfbff6f499fa97}{main()}} functions to be called.
\end{DoxyItemize}
\item What doesn\textquotesingle{}t work\+:
\begin{DoxyItemize}
\item compilation with anything other than Visual C++
\item programmatically-\/created custom cursors. These don\textquotesingle{}t appear to be supported by Win\+RT. Different O\+S-\/provided cursors can, however, be created via \mbox{\hyperlink{_s_d_l__mouse_8h_a217c89743b3da84b9656304f8e4ca51e}{S\+D\+L\+\_\+\+Create\+System\+Cursor()}} (unsupported on Windows Phone)
\item \mbox{\hyperlink{_s_d_l__mouse_8h_a2f2259cdf30272b3c4cc1e3236cb8071}{S\+D\+L\+\_\+\+Warp\+Mouse\+In\+Window()}} or \mbox{\hyperlink{_s_d_l__mouse_8h_a200c3d105a0a30fb465760ab870b3ce3}{S\+D\+L\+\_\+\+Warp\+Mouse\+Global()}}. This are not currently supported by Win\+RT itself.
\item joysticks and game controllers that either are not supported by Microsoft\textquotesingle{}s X\+Input A\+PI, or are not supported within U\+WP apps (many controllers that work in Win32, do not work in U\+WP, due to restrictions in U\+WP itself).
\item turning off V\+Sync when rendering on Windows Phone. Attempts to turn V\+Sync off on Windows Phone result either in Direct3D not drawing anything, or it forcing V\+Sync back on. As such, S\+D\+L\+\_\+\+R\+E\+N\+D\+E\+R\+E\+R\+\_\+\+P\+R\+E\+S\+E\+N\+T\+V\+S\+Y\+NC will always get turned-\/on on Windows Phone. This limitation is not present in non-\/\+Phone Win\+RT (such as Windows 8.\+x), where turning off V\+Sync appears to work.
\item probably anything else that\textquotesingle{}s not listed as supported
\end{DoxyItemize}
\end{DoxyItemize}

\subsection*{Upgrade Notes }

\paragraph*{\mbox{\hyperlink{_s_d_l__filesystem_8h_ab11eaf74d913eefb472475f0c8e312ce}{S\+D\+L\+\_\+\+Get\+Pref\+Path()}} usage when upgrading Win\+RT apps from S\+DL 2.\+0.\+3}

S\+DL 2.\+0.\+4 fixes two bugs found in the Win\+RT version of \mbox{\hyperlink{_s_d_l__filesystem_8h_ab11eaf74d913eefb472475f0c8e312ce}{S\+D\+L\+\_\+\+Get\+Pref\+Path()}}. The fixes may affect older, S\+DL 2.\+0.\+3-\/based apps\textquotesingle{} save data. Please note that these changes only apply to S\+D\+L-\/based Win\+RT apps, and not to apps for any other platform.


\begin{DoxyEnumerate}
\item \mbox{\hyperlink{_s_d_l__filesystem_8h_ab11eaf74d913eefb472475f0c8e312ce}{S\+D\+L\+\_\+\+Get\+Pref\+Path()}} would return an invalid path, one in which the path\textquotesingle{}s directory had not been created. Attempts to create files there (via fopen(), for example), would fail, unless that directory was explicitly created beforehand.
\item \mbox{\hyperlink{_s_d_l__filesystem_8h_ab11eaf74d913eefb472475f0c8e312ce}{S\+D\+L\+\_\+\+Get\+Pref\+Path()}}, for non-\/\+Win\+Phone-\/based apps, would return a path inside a Win\+RT \textquotesingle{}Roaming\textquotesingle{} folder, the contents of which get automatically synchronized across multiple devices. This process can occur while an application runs, and can cause existing save-\/data to be overwritten at unexpected times, with data from other devices. (Windows Phone apps written with S\+DL 2.\+0.\+3 did not utilize a Roaming folder, due to A\+PI restrictions in Windows Phone 8.\+0).
\end{DoxyEnumerate}

\mbox{\hyperlink{_s_d_l__filesystem_8h_ab11eaf74d913eefb472475f0c8e312ce}{S\+D\+L\+\_\+\+Get\+Pref\+Path()}}, starting with S\+DL 2.\+0.\+4, addresses these by\+:


\begin{DoxyEnumerate}
\item making sure that \mbox{\hyperlink{_s_d_l__filesystem_8h_ab11eaf74d913eefb472475f0c8e312ce}{S\+D\+L\+\_\+\+Get\+Pref\+Path()}} returns a directory in which data can be written to immediately, without first needing to create directories.
\item basing \mbox{\hyperlink{_s_d_l__filesystem_8h_ab11eaf74d913eefb472475f0c8e312ce}{S\+D\+L\+\_\+\+Get\+Pref\+Path()}} off of a different, non-\/\+Roaming folder, the contents of which do not automatically get synchronized across devices (and which require less work to use safely, in terms of data integrity).
\end{DoxyEnumerate}

Apps that wish to get their Roaming folder\textquotesingle{}s path can do so either by using S\+D\+L\+\_\+\+Win\+R\+T\+Get\+F\+S\+Path\+U\+T\+F8(), S\+D\+L\+\_\+\+Win\+R\+T\+Get\+F\+S\+Path\+U\+N\+I\+C\+O\+D\+E() (which returns a U\+C\+S-\/2/wide-\/char string), or directly through the Win\+RT class, Windows.\+Storage.\+Application\+Data.

\subsection*{Setup, High-\/\+Level Steps }

The steps for setting up a project for an S\+D\+L/\+Win\+RT app looks like the following, at a high-\/level\+:


\begin{DoxyEnumerate}
\item create a new Visual C++ project using Microsoft\textquotesingle{}s template for a, \char`\"{}\+Direct3\+D App\char`\"{}.
\item remove most of the files from the project.
\item make your app\textquotesingle{}s project directly reference S\+D\+L/\+Win\+RT\textquotesingle{}s own Visual C++ project file, via use of Visual C++\textquotesingle{}s \char`\"{}\+References\char`\"{} dialog. This will setup the linker, and will copy S\+DL\textquotesingle{}s .dll files to your app\textquotesingle{}s final output.
\item adjust your app\textquotesingle{}s build settings, at minimum, telling it where to find S\+DL\textquotesingle{}s header files.
\item add files that contains a Win\+R\+T-\/appropriate main function, along with some data to make sure mouse-\/cursor-\/hiding (via S\+D\+L\+\_\+\+Show\+Cursor(\+S\+D\+L\+\_\+\+D\+I\+S\+A\+B\+L\+E) calls) work properly.
\item add S\+D\+L-\/specific app code.
\item build and run your app.
\end{DoxyEnumerate}

\subsection*{Setup, Detailed Steps }

\subsubsection*{1. Create a new project}

Create a new project using one of Visual C++\textquotesingle{}s templates for a plain, non-\/\+X\+A\+ML, \char`\"{}\+Direct3\+D App\char`\"{} (X\+A\+ML support for S\+D\+L/\+Win\+RT is not yet ready for use). If you don\textquotesingle{}t see one of these templates, in Visual C++\textquotesingle{}s \textquotesingle{}New Project\textquotesingle{} dialog, try using the textbox titled, \textquotesingle{}Search Installed Templates\textquotesingle{} to look for one.

\subsubsection*{2. Remove unneeded files from the project}

In the new project, delete any file that has one of the following extensions\+:


\begin{DoxyItemize}
\item .cpp
\item .h
\item .hlsl
\end{DoxyItemize}

When you are done, you should be left with a few files, each of which will be a necessary part of your app\textquotesingle{}s project. These files will consist of\+:


\begin{DoxyItemize}
\item an .appxmanifest file, which contains metadata on your Win\+RT app. This is similar to an Info.\+plist file on i\+OS, or an Android\+Manifest.\+xml on Android.
\item a few .png files, one of which is a splash screen (displayed when your app launches), others are app icons.
\item a .pfx file, used for code signing purposes.
\end{DoxyItemize}

\subsubsection*{3. Add references to S\+DL\textquotesingle{}s project files}

S\+D\+L/\+Win\+RT can be built in multiple variations, spanning across three different C\+PU architectures (x86, x64, and A\+RM) and two different configurations (Debug and Release). Win\+RT and Visual C++ do not currently provide a means for combining multiple variations of one library into a single file. Furthermore, it does not provide an easy means for copying pre-\/built .dll files into your app\textquotesingle{}s final output (via Post-\/\+Build steps, for example). It does, however, provide a system whereby an app can reference the M\+S\+VC projects of libraries such that, when the app is built\+:


\begin{DoxyEnumerate}
\item each library gets built for the appropriate C\+PU architecture(s) and Win\+RT platform(s).
\item each library\textquotesingle{}s output, such as .dll files, get copied to the app\textquotesingle{}s build output.
\end{DoxyEnumerate}

To set this up for S\+D\+L/\+Win\+RT, you\textquotesingle{}ll need to run through the following steps\+:


\begin{DoxyEnumerate}
\item open up the Solution Explorer inside Visual C++ (under the \char`\"{}\+View\char`\"{} menu, then \char`\"{}\+Solution Explorer\char`\"{})
\item right click on your app\textquotesingle{}s solution.
\item navigate to \char`\"{}\+Add\char`\"{}, then to \char`\"{}\+Existing Project...\char`\"{}
\item find S\+D\+L/\+Win\+RT\textquotesingle{}s Visual C++ project file and open it. Different project files exist for different Win\+RT platforms. All of them are in S\+DL\textquotesingle{}s source distribution, in the following directories\+:
\begin{DoxyItemize}
\item {\ttfamily Visual\+C-\/\+Win\+R\+T/\+U\+W\+P\+\_\+\+V\+S2015/} -\/ for Windows 10 / U\+WP apps
\item {\ttfamily Visual\+C-\/\+Win\+R\+T/\+Win\+Phone81\+\_\+\+V\+S2013/} -\/ for Windows Phone 8.\+1 apps
\item {\ttfamily Visual\+C-\/\+Win\+R\+T/\+Win\+R\+T80\+\_\+\+V\+S2012/} -\/ for Windows 8.\+0 apps
\item {\ttfamily Visual\+C-\/\+Win\+R\+T/\+Win\+R\+T81\+\_\+\+V\+S2013/} -\/ for Windows 8.\+1 apps
\end{DoxyItemize}
\item once the project has been added, right-\/click on your app\textquotesingle{}s project and select, \char`\"{}\+References...\char`\"{}
\item click on the button titled, \char`\"{}\+Add New Reference...\char`\"{}
\item check the box next to S\+DL
\item click OK to close the dialog
\item S\+DL will now show up in the list of references. Click OK to close that dialog.
\end{DoxyEnumerate}

Your project is now linked to S\+DL\textquotesingle{}s project, insofar that when the app is built, S\+DL will be built as well, with its build output getting included with your app.

\subsubsection*{4. Adjust Your App\textquotesingle{}s Build Settings}

Some build settings need to be changed in your app\textquotesingle{}s project. This guide will outline the following\+:


\begin{DoxyItemize}
\item making sure that the compiler knows where to find S\+DL\textquotesingle{}s header files
\item {\bfseries Optional for C++, but N\+E\+C\+E\+S\+S\+A\+RY for compiling C code\+:} telling the compiler not to use Microsoft\textquotesingle{}s C++ extensions for Win\+RT development.
\item {\bfseries Optional\+:} telling the compiler not generate errors due to missing precompiled header files.
\end{DoxyItemize}

To change these settings\+:


\begin{DoxyEnumerate}
\item right-\/click on the project
\item choose \char`\"{}\+Properties\char`\"{}
\item in the drop-\/down box next to \char`\"{}\+Configuration\char`\"{}, choose, \char`\"{}\+All Configurations\char`\"{}
\item in the drop-\/down box next to \char`\"{}\+Platform\char`\"{}, choose, \char`\"{}\+All Platforms\char`\"{}
\item in the left-\/hand list, expand the \char`\"{}\+C/\+C++\char`\"{} section
\item select \char`\"{}\+General\char`\"{}
\item edit the \char`\"{}\+Additional Include Directories\char`\"{} setting, and add a path to S\+DL\textquotesingle{}s \char`\"{}include\char`\"{} directory
\item {\bfseries Optional\+: to enable compilation of C code\+:} change the setting for \char`\"{}\+Consume Windows Runtime Extension\char`\"{} from \char`\"{}\+Yes (/\+Z\+W)\char`\"{} to \char`\"{}\+No\char`\"{}. If you\textquotesingle{}re working with a completely C++ based project, this step can usually be omitted.
\item {\bfseries Optional\+: to disable precompiled headers (which can produce \textquotesingle{}stdafx.\+h\textquotesingle{}-\/related build errors, if setup incorrectly\+:} in the left-\/hand list, select \char`\"{}\+Precompiled Headers\char`\"{}, then change the setting for \char`\"{}\+Precompiled 
   Header\char`\"{} from \char`\"{}\+Use (/\+Yu)\char`\"{} to \char`\"{}\+Not Using Precompiled Headers\char`\"{}.
\item close the dialog, saving settings, by clicking the \char`\"{}\+O\+K\char`\"{} button
\end{DoxyEnumerate}

\subsubsection*{5. Add a Win\+R\+T-\/appropriate main function, and a blank-\/cursor image, to the app.}

A few files should be included directly in your app\textquotesingle{}s M\+S\+VC project, specifically\+:
\begin{DoxyEnumerate}
\item a Win\+R\+T-\/appropriate main function (which is different than \mbox{\hyperlink{_c_make_c_compiler_id_8c_a0ddf1224851353fc92bfbff6f499fa97}{main()}} functions on other platforms)
\item a Win32-\/style cursor resource, used by \mbox{\hyperlink{_s_d_l__mouse_8h_a00286ec15cd56dee1fd71ed4e6e7a585}{S\+D\+L\+\_\+\+Show\+Cursor()}} to hide the mouse cursor (if and when the app needs to do so). {\itshape If this cursor resource is not included, mouse-\/position reporting may fail if and when the cursor is hidden, due to possible bugs/design-\/oddities in Windows itself.}
\end{DoxyEnumerate}

To include these files\+:


\begin{DoxyEnumerate}
\item right-\/click on your project (again, in Visual C++\textquotesingle{}s Solution Explorer), navigate to \char`\"{}\+Add\char`\"{}, then choose \char`\"{}\+Existing Item...\char`\"{}.
\item navigate to the directory containing S\+DL\textquotesingle{}s source code, then into its subdirectory, \textquotesingle{}src/main/winrt/\textquotesingle{}. Select, then add, the following files\+:
\begin{DoxyItemize}
\item {\ttfamily S\+D\+L\+\_\+winrt\+\_\+main\+\_\+\+Non\+X\+A\+M\+L.\+cpp}
\item {\ttfamily S\+D\+L2-\/\+Win\+R\+T\+Resources.\+rc}
\item {\ttfamily S\+D\+L2-\/\+Win\+R\+T\+Resource\+\_\+\+Blank\+Cursor.\+cur}
\end{DoxyItemize}
\item right-\/click on the file {\ttfamily S\+D\+L\+\_\+winrt\+\_\+main\+\_\+\+Non\+X\+A\+M\+L.\+cpp} (as listed in your project), then click on \char`\"{}\+Properties...\char`\"{}.
\item in the drop-\/down box next to \char`\"{}\+Configuration\char`\"{}, choose, \char`\"{}\+All Configurations\char`\"{}
\item in the drop-\/down box next to \char`\"{}\+Platform\char`\"{}, choose, \char`\"{}\+All Platforms\char`\"{}
\item in the left-\/hand list, click on \char`\"{}\+C/\+C++\char`\"{}
\item change the setting for \char`\"{}\+Consume Windows Runtime Extension\char`\"{} to \char`\"{}\+Yes (/\+Z\+W)\char`\"{}.
\item click the OK button. This will close the dialog.
\end{DoxyEnumerate}

{\bfseries N\+O\+TE\+: C++/\+CX compilation is currently required in at least one file of your app\textquotesingle{}s project. This is to make sure that Visual C++\textquotesingle{}s linker builds a \textquotesingle{}Windows Metadata\textquotesingle{} file (.winmd) for your app. Not doing so can lead to build errors.}

\subsubsection*{6. Add app code and assets}

At this point, you can add in S\+D\+L-\/specific source code. Be sure to include a C-\/style main function (ie\+: {\ttfamily int \mbox{\hyperlink{main_8cpp_a0ddf1224851353fc92bfbff6f499fa97}{main(int argc, char $\ast$argv\mbox{[}$\,$\mbox{]})}}}). From there you should be able to create a single {\ttfamily S\+D\+L\+\_\+\+Window} (Win\+RT apps can only have one window, at present), as well as an {\ttfamily S\+D\+L\+\_\+\+Renderer}. Direct3D will be used to draw content. Events are received via S\+DL\textquotesingle{}s usual event functions ({\ttfamily S\+D\+L\+\_\+\+Poll\+Event}, etc.) If you have a set of existing source files and assets, you can start adding them to the project now. If not, or if you would like to make sure that you\textquotesingle{}re setup correctly, some short and simple sample code is provided below.

\paragraph*{6.\+A. ... when creating a new app}

If you are creating a new app (rather than porting an existing S\+D\+L-\/based app), or if you would just like a simple app to test S\+D\+L/\+Win\+RT with before trying to get existing code working, some working S\+D\+L/\+Win\+RT code is provided below. To set this up\+:


\begin{DoxyEnumerate}
\item right click on your app\textquotesingle{}s project
\item select Add, then New Item. An \char`\"{}\+Add New Item\char`\"{} dialog will show up.
\item from the left-\/hand list, choose \char`\"{}\+Visual C++\char`\"{}
\item from the middle/main list, choose \char`\"{}\+C++ File (.\+cpp)\char`\"{}
\item near the bottom of the dialog, next to \char`\"{}\+Name\+:\char`\"{}, type in a name for your source file, such as, \char`\"{}main.\+cpp\char`\"{}.
\item click on the Add button. This will close the dialog, add the new file to your project, and open the file in Visual C++\textquotesingle{}s text editor.
\item Copy and paste the following code into the new file, then save it.
\end{DoxyEnumerate}

\begin{DoxyVerb}#include <SDL.h>

int main(int argc, char **argv)
{
    SDL_DisplayMode mode;
    SDL_Window * window = NULL;
    SDL_Renderer * renderer = NULL;
    SDL_Event evt;

    if (SDL_Init(SDL_INIT_VIDEO) != 0) {
        return 1;
    }

    if (SDL_GetCurrentDisplayMode(0, &mode) != 0) {
        return 1;
    }

    if (SDL_CreateWindowAndRenderer(mode.w, mode.h, SDL_WINDOW_FULLSCREEN, &window, &renderer) != 0) {
        return 1;
    }

    while (1) {
        while (SDL_PollEvent(&evt)) {
        }

        SDL_SetRenderDrawColor(renderer, 0, 255, 0, 255);
        SDL_RenderClear(renderer);
        SDL_RenderPresent(renderer);
    }
}
\end{DoxyVerb}


\paragraph*{6.\+B. Adding code and assets}

If you have existing code and assets that you\textquotesingle{}d like to add, you should be able to add them now. The process for adding a set of files is as such.


\begin{DoxyEnumerate}
\item right click on the app\textquotesingle{}s project
\item select Add, then click on \char`\"{}\+New Item...\char`\"{}
\item open any source, header, or asset files as appropriate. Support for C and C++ is available.
\end{DoxyEnumerate}

Do note that Win\+RT only supports a subset of the A\+P\+Is that are available to Win32-\/based apps. Many portions of the Win32 A\+PI and the C runtime are not available.

A list of unsupported C A\+P\+Is can be found at \href{http://msdn.microsoft.com/en-us/library/windows/apps/jj606124.aspx}{\tt http\+://msdn.\+microsoft.\+com/en-\/us/library/windows/apps/jj606124.\+aspx}

General information on using the C runtime in Win\+RT can be found at \href{https://msdn.microsoft.com/en-us/library/hh972425.aspx}{\tt https\+://msdn.\+microsoft.\+com/en-\/us/library/hh972425.\+aspx}

A list of supported Win32 A\+P\+Is for Win\+RT apps can be found at \href{http://msdn.microsoft.com/en-us/library/windows/apps/br205757.aspx}{\tt http\+://msdn.\+microsoft.\+com/en-\/us/library/windows/apps/br205757.\+aspx}. To note, the list of supported Win32 A\+P\+Is for Windows Phone 8.\+0 is different. ~\newline
That list can be found at \href{http://msdn.microsoft.com/en-us/library/windowsphone/develop/jj662956(v=vs.105).aspx}{\tt http\+://msdn.\+microsoft.\+com/en-\/us/library/windowsphone/develop/jj662956(v=vs.\+105).\+aspx}

\subsubsection*{7. Build and run your app}

Your app project should now be setup, and you should be ready to build your app. ~\newline
To run it on the local machine, open the Debug menu and choose \char`\"{}\+Start 
\+Debugging\char`\"{}. This will build your app, then run your app full-\/screen. To switch out of your app, press the Windows key. Alternatively, you can choose to run your app in a window. To do this, before building and running your app, find the drop-\/down menu in Visual C++\textquotesingle{}s toolbar that says, \char`\"{}\+Local Machine\char`\"{}. Expand this by clicking on the arrow on the right side of the list, then click on Simulator. Once you do that, any time you build and run the app, the app will launch in window, rather than full-\/screen.

\paragraph*{7.\+A. Running apps on older, A\+R\+M-\/based, \char`\"{}\+Windows R\+T\char`\"{} devices}

{\bfseries These instructions do not include Windows Phone, despite Windows Phone typically running on A\+RM processors.} They are specifically for devices that use the \char`\"{}\+Windows R\+T\char`\"{} operating system, which was a modified version of Windows 8.\+x that ran primarily on A\+R\+M-\/based tablet computers.

To build and run the app on A\+R\+M-\/based, \char`\"{}\+Windows R\+T\char`\"{} devices, you\textquotesingle{}ll need to\+:


\begin{DoxyItemize}
\item install Microsoft\textquotesingle{}s \char`\"{}\+Remote Debugger\char`\"{} on the device. Visual C++ installs and debugs A\+R\+M-\/based apps via IP networks.
\item change a few options on the development machine, both to make sure it builds for A\+RM (rather than x86 or x64), and to make sure it knows how to find the Windows RT device (on the network).
\end{DoxyItemize}

Microsoft\textquotesingle{}s Remote Debugger can be found at \href{https://msdn.microsoft.com/en-us/library/hh441469.aspx}{\tt https\+://msdn.\+microsoft.\+com/en-\/us/library/hh441469.\+aspx}. Please note that separate versions of this debugger exist for different versions of Visual C++, one each for M\+S\+VC 2015, 2013, and 2012.

To setup Visual C++ to launch your app on an A\+RM device\+:


\begin{DoxyEnumerate}
\item make sure the Remote Debugger is running on your A\+RM device, and that it\textquotesingle{}s on the same IP network as your development machine.
\item from Visual C++\textquotesingle{}s toolbar, find a drop-\/down menu that says, \char`\"{}\+Win32\char`\"{}. Click it, then change the value to \char`\"{}\+A\+R\+M\char`\"{}.
\item make sure Visual C++ knows the hostname or IP address of the A\+RM device. To do this\+:
\begin{DoxyEnumerate}
\item open the app project\textquotesingle{}s properties
\item select \char`\"{}\+Debugging\char`\"{}
\item next to \char`\"{}\+Machine Name\char`\"{}, enter the hostname or IP address of the A\+RM device
\item if, and only if, you\textquotesingle{}ve turned off authentication in the Remote Debugger, then change the setting for \char`\"{}\+Require Authentication\char`\"{} to No
\item click \char`\"{}\+O\+K\char`\"{}
\end{DoxyEnumerate}
\item build and run the app (from Visual C++). The first time you do this, a prompt will show up on the A\+RM device, asking for a Microsoft Account. You do, unfortunately, need to log in here, and will need to follow the subsequent registration steps in order to launch the app. After you do so, if the app didn\textquotesingle{}t already launch, try relaunching it again from within Visual C++.
\end{DoxyEnumerate}

\subsection*{Troubleshooting }

\paragraph*{Build fails with message, \char`\"{}error L\+N\+K2038\+: mismatch detected for \textquotesingle{}vccorlib\+\_\+lib\+\_\+should\+\_\+be\+\_\+specified\+\_\+before\+\_\+msvcrt\+\_\+lib\+\_\+to\+\_\+linker\textquotesingle{}\char`\"{}}

Try adding the following to your linker flags. In M\+S\+VC, this can be done by right-\/clicking on the app project, navigating to Configuration Properties -\/$>$ Linker -\/$>$ Command Line, then adding them to the Additional Options section.


\begin{DoxyItemize}
\item For Release builds / M\+S\+V\+C-\/\+Configurations, add\+:

/nodefaultlib\+:vccorlib /nodefaultlib\+:msvcrt vccorlib.\+lib msvcrt.\+lib
\item For Debug builds / M\+S\+V\+C-\/\+Configurations, add\+:

/nodefaultlib\+:vccorlibd /nodefaultlib\+:msvcrtd vccorlibd.\+lib msvcrtd.\+lib
\end{DoxyItemize}

\paragraph*{Mouse-\/motion events fail to get sent, or \mbox{\hyperlink{_s_d_l__mouse_8h_a1561f413546c0e4f5f44a8f094926575}{S\+D\+L\+\_\+\+Get\+Mouse\+State()}} fails to return updated values}

This may be caused by a bug in Windows itself, whereby hiding the mouse cursor can cause mouse-\/position reporting to fail.

S\+DL provides a workaround for this, but it requires that an app links to a set of Win32-\/style cursor image-\/resource files. A copy of suitable resource files can be found in {\ttfamily src/main/winrt/}. Adding them to an app\textquotesingle{}s Visual C++ project file should be sufficient to get the app to use them.

\paragraph*{S\+DL\textquotesingle{}s Visual Studio project file fails to open, with message, \char`\"{}\+The system can\textquotesingle{}t find the file specified.\char`\"{}}

This can be caused for any one of a few reasons, which Visual Studio can report, but won\textquotesingle{}t always do so in an up-\/front manner.

To help determine why this error comes up\+:


\begin{DoxyEnumerate}
\item open a copy of Visual Studio without opening a project file. This can be accomplished via Windows\textquotesingle{} Start Menu, among other means.
\item show Visual Studio\textquotesingle{}s Output window. This can be done by going to VS\textquotesingle{} menu bar, then to View, and then to Output.
\item try opening the S\+DL project file directly by going to VS\textquotesingle{} menu bar, then to File, then to Open, then to Project/\+Solution. When a File-\/\+Open dialog appears, open the S\+DL project (such as the one in S\+DL\textquotesingle{}s source code, in its directory, Visual\+C-\/\+Win\+R\+T/\+U\+W\+P\+\_\+\+V\+S2015/).
\item after attempting to open S\+DL\textquotesingle{}s Visual Studio project file, additional error information will be output to the Output window.
\end{DoxyEnumerate}

If Visual Studio reports (via its Output window) that the project\+:

\char`\"{}could not be loaded because it\textquotesingle{}s missing install components. To fix this launch Visual Studio setup with the following selections\+:
\+Microsoft.\+Visual\+Studio.\+Component\+Group.\+U\+W\+P.\+V\+C\char`\"{}

... then you will need to re-\/launch Visual Studio\textquotesingle{}s installer, and make sure that the workflow for \char`\"{}\+Universal Windows Platform development\char`\"{} is checked, and that its optional component, \char`\"{}\+C++ Universal Windows Platform tools\char`\"{} is also checked. While you are there, if you are planning on targeting U\+WP / Windows 10, also make sure that you check the optional component, \char`\"{}\+Windows 10 S\+D\+K (10.\+0.\+10240.\+0)\char`\"{}. After making sure these items are checked as-\/appropriate, install them.

Once you install these components, try re-\/launching Visual Studio, and re-\/opening the S\+DL project file. If you still get the error dialog, try using the Output window, again, seeing what Visual Studio says about it.

\paragraph*{Game controllers / joysticks aren\textquotesingle{}t working!}

Windows only permits certain game controllers and joysticks to work within Win\+RT / U\+WP apps. Even if a game controller or joystick works in a Win32 app, that device is not guaranteed to work inside a Win\+RT / U\+WP app.

According to Microsoft, \char`\"{}\+Xbox compatible controllers\char`\"{} should work inside U\+WP apps, potentially with more working in the future. This includes, but may not be limited to, Microsoft-\/made Xbox controllers and U\+SB adapters. (Source\+: \href{https://social.msdn.microsoft.com/Forums/en-US/9064838b-e8c3-4c18-8a83-19bf0dfe150d/xinput-fails-to-detect-game-controllers?forum=wpdevelop}{\tt https\+://social.\+msdn.\+microsoft.\+com/\+Forums/en-\/\+U\+S/9064838b-\/e8c3-\/4c18-\/8a83-\/19bf0dfe150d/xinput-\/fails-\/to-\/detect-\/game-\/controllers?forum=wpdevelop}) 