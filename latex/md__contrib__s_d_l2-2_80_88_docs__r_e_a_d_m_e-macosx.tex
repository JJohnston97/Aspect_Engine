These instructions are for people using Apple\textquotesingle{}s Mac OS X (pronounced \char`\"{}ten\char`\"{}).

From the developer\textquotesingle{}s point of view, OS X is a sort of hybrid Mac and Unix system, and you have the option of using either traditional command line tools or Apple\textquotesingle{}s I\+DE Xcode.

\section*{Command Line Build }

To build S\+DL using the command line, use the standard configure and make process\+: \begin{DoxyVerb}./configure
make
sudo make install
\end{DoxyVerb}


You can also build S\+DL as a Universal library (a single binary for both 32-\/bit and 64-\/bit Intel architectures), on Mac OS X 10.\+7 and newer, by using the gcc-\/fat.\+sh script in build-\/scripts\+: \begin{DoxyVerb}mkdir mybuild
cd mybuild
CC=$PWD/../build-scripts/gcc-fat.sh CXX=$PWD/../build-scripts/g++-fat.sh ../configure
make
sudo make install
\end{DoxyVerb}


This script builds S\+DL with 10.\+5 A\+BI compatibility on i386 and 10.\+6 A\+BI compatibility on x86\+\_\+64 architectures. For best compatibility you should compile your application the same way.

Please note that building S\+DL requires at least Xcode 4.\+6 and the 10.\+7 S\+DK (even if you target back to 10.\+5 systems). Power\+PC support for Mac OS X has been officially dropped as of S\+DL 2.\+0.\+2.

To use the library once it\textquotesingle{}s built, you essential have two possibilities\+: use the traditional autoconf/automake/make method, or use Xcode. 

 \section*{Caveats for using S\+DL with Mac OS X }

Some things you have to be aware of when using S\+DL on Mac OS X\+:


\begin{DoxyItemize}
\item If you register your own N\+S\+Application\+Delegate (using \mbox{[}N\+S\+App set\+Delegate\+:\mbox{]}), S\+DL will not register its own. This means that S\+DL will not terminate using S\+D\+L\+\_\+\+Quit if it receives a termination request, it will terminate like a normal app, and it will not send a S\+D\+L\+\_\+\+D\+R\+O\+P\+F\+I\+LE when you request to open a file with the app. To solve these issues, put the following code in your N\+S\+Application\+Delegate implementation\+:
\end{DoxyItemize}

\begin{DoxyVerb}- (NSApplicationTerminateReply)applicationShouldTerminate:(NSApplication *)sender
{
    if (SDL_GetEventState(SDL_QUIT) == SDL_ENABLE) {
        SDL_Event event;
        event.type = SDL_QUIT;
        SDL_PushEvent(&event);
    }

    return NSTerminateCancel;
}

- (BOOL)application:(NSApplication *)theApplication openFile:(NSString *)filename
{
    if (SDL_GetEventState(SDL_DROPFILE) == SDL_ENABLE) {
        SDL_Event event;
        event.type = SDL_DROPFILE;
        event.drop.file = SDL_strdup([filename UTF8String]);
        return (SDL_PushEvent(&event) > 0);
    }

    return NO;
}
\end{DoxyVerb}


============================================================================== \section*{Using the Simple Direct\+Media Layer with a traditional Makefile }

An existing autoconf/automake build system for your S\+DL app has good chances to work almost unchanged on OS X. However, to produce a \char`\"{}real\char`\"{} Mac OS X binary that you can distribute to users, you need to put the generated binary into a so called \char`\"{}bundle\char`\"{}, which basically is a fancy folder with a name like \char`\"{}\+My\+Cool\+Game.\+app\char`\"{}.

To get this build automatically, add something like the following rule to your Makefile.\+am\+: \begin{DoxyVerb}bundle_contents = APP_NAME.app/Contents
APP_NAME_bundle: EXE_NAME
    mkdir -p $(bundle_contents)/MacOS
    mkdir -p $(bundle_contents)/Resources
    echo "APPL????" > $(bundle_contents)/PkgInfo
    $(INSTALL_PROGRAM) $< $(bundle_contents)/MacOS/
\end{DoxyVerb}


You should replace E\+X\+E\+\_\+\+N\+A\+ME with the name of the executable. A\+P\+P\+\_\+\+N\+A\+ME is what will be visible to the user in the Finder. Usually it will be the same as E\+X\+E\+\_\+\+N\+A\+ME but capitalized. E.\+g. if E\+X\+E\+\_\+\+N\+A\+ME is \char`\"{}testgame\char`\"{} then A\+P\+P\+\_\+\+N\+A\+ME usually is \char`\"{}\+Test\+Game\char`\"{}. You might also want to use {\ttfamily @P\+A\+C\+K\+A\+GE@} to use the package name as specified in your configure.\+in file.

If your project builds more than one application, you will have to do a bit more. For each of your target applications, you need a separate rule.

If you want the created bundles to be installed, you may want to add this rule to your Makefile.\+am\+: \begin{DoxyVerb}install-exec-hook: APP_NAME_bundle
    rm -rf $(DESTDIR)$(prefix)/Applications/APP_NAME.app
    mkdir -p $(DESTDIR)$(prefix)/Applications/
    cp -r $< /$(DESTDIR)$(prefix)Applications/
\end{DoxyVerb}


This rule takes the Bundle created by the rule from step 3 and installs them into \char`\"{}\$(\+D\+E\+S\+T\+D\+I\+R)\$(prefix)/\+Applications/\char`\"{}.

Again, if you want to install multiple applications, you will have to augment the make rule accordingly.

But beware! That is only part of the story! With the above, you end up with a bare bone .app bundle, which is double clickable from the Finder. But there are some more things you should do before shipping your product...

1) The bundle right now probably is dynamically linked against S\+DL. That means that when you copy it to another computer, {\itshape it will not run}, unless you also install S\+DL on that other computer. A good solution for this dilemma is to static link against S\+DL. On OS X, you can achieve that by linking against the libraries listed by \begin{DoxyVerb}sdl-config --static-libs
\end{DoxyVerb}


instead of those listed by \begin{DoxyVerb}sdl-config --libs
\end{DoxyVerb}


Depending on how exactly S\+DL is integrated into your build systems, the way to achieve that varies, so I won\textquotesingle{}t describe it here in detail

2) Add an \textquotesingle{}Info.\+plist\textquotesingle{} to your application. That is a special X\+ML file which contains some meta-\/information about your application (like some copyright information, the version of your app, the name of an optional icon file, and other things). Part of that information is displayed by the Finder when you click on the .app, or if you look at the \char`\"{}\+Get Info\char`\"{} window. More information about Info.\+plist files can be found on Apple\textquotesingle{}s homepage.

As a final remark, let me add that I use some of the techniques (and some variations of them) in Exult and Scumm\+VM; both are available in source on the net, so feel free to take a peek at them for inspiration!



 \section*{Using the Simple Direct\+Media Layer with Xcode }

These instructions are for using Apple\textquotesingle{}s Xcode I\+DE to build S\+DL applications.


\begin{DoxyItemize}
\item First steps
\end{DoxyItemize}

The first thing to do is to unpack the Xcode.\+tar.\+gz archive in the top level S\+DL directory (where the Xcode.\+tar.\+gz archive resides). Because Stuffit Expander will unpack the archive into a subdirectory, you should unpack the archive manually from the command line\+: \begin{DoxyVerb}cd [path_to_SDL_source]
tar zxf Xcode.tar.gz
\end{DoxyVerb}


This will create a new folder called Xcode, which you can browse normally from the Finder.


\begin{DoxyItemize}
\item Building the Framework
\end{DoxyItemize}

The S\+DL Library is packaged as a framework bundle, an organized relocatable folder hierarchy of executable code, interface headers, and additional resources. For practical purposes, you can think of a framework as a more user and system-\/friendly shared library, whose library file behaves more or less like a standard U\+N\+IX shared library.

To build the framework, simply open the framework project and build it. By default, the framework bundle \char`\"{}\+S\+D\+L.\+framework\char`\"{} is installed in /\+Library/\+Frameworks. Therefore, the testers and project stationary expect it to be located there. However, it will function the same in any of the following locations\+: \begin{DoxyVerb}~/Library/Frameworks
/Local/Library/Frameworks
/System/Library/Frameworks
\end{DoxyVerb}



\begin{DoxyItemize}
\item Build Options There are two \char`\"{}\+Build Styles\char`\"{} (See the \char`\"{}\+Targets\char`\"{} tab) for S\+DL. \char`\"{}\+Deployment\char`\"{} should be used if you aren\textquotesingle{}t tweaking the S\+DL library. \char`\"{}\+Development\char`\"{} should be used to debug S\+DL apps or the library itself.
\item Building the Testers Open the S\+D\+L\+Test project and build away!
\item Using the Project Stationary Copy the stationary to the indicated folders to access it from the \char`\"{}\+New Project\char`\"{} and \char`\"{}\+Add target\char`\"{} menus. What could be easier?
\item Setting up a new project by hand Some of you won\textquotesingle{}t want to use the Stationary so I\textquotesingle{}ll give some tips\+:
\begin{DoxyItemize}
\item Create a new \char`\"{}\+Cocoa Application\char`\"{}
\item Add src/main/macosx/\+S\+D\+L\+Main.\+m , .h and .nib to your project
\item Remove \char`\"{}main.\+c\char`\"{} from your project
\item Remove \char`\"{}\+Main\+Menu.\+nib\char`\"{} from your project
\item Add \char`\"{}\$(\+H\+O\+M\+E)/\+Library/\+Frameworks/\+S\+D\+L.\+framework/\+Headers\char`\"{} to include path
\item Add \char`\"{}\$(\+H\+O\+M\+E)/\+Library/\+Frameworks\char`\"{} to the frameworks search path
\item Add \char`\"{}-\/framework S\+D\+L -\/framework Foundation -\/framework App\+Kit\char`\"{} to \char`\"{}\+O\+T\+H\+E\+R\+\_\+\+L\+D\+F\+L\+A\+G\+S\char`\"{}
\item Set the \char`\"{}\+Main Nib File\char`\"{} under \char`\"{}\+Application Settings\char`\"{} to \char`\"{}\+S\+D\+L\+Main.\+nib\char`\"{}
\item Add your files
\item Clean and build
\end{DoxyItemize}
\item Building from command line Use pbxbuild in the same directory as your .pbproj file
\item Running your app You can send command line args to your app by either invoking it from the command line (in $\ast$.app/\+Contents/\+Mac\+OS) or by entering them in the \char`\"{}\+Executables\char`\"{} panel of the target settings.
\item Implementation Notes Some things that may be of interest about how it all works...
\begin{DoxyItemize}
\item Working directory As defined in the S\+D\+L\+\_\+main.\+m file, the working directory of your S\+DL app is by default set to its parent. You may wish to change this to better suit your needs.
\item You have a Cocoa App! Your S\+DL app is essentially a Cocoa application. When your app starts up and the libraries finish loading, a Cocoa procedure is called, which sets up the working directory and calls your \mbox{\hyperlink{_c_make_c_compiler_id_8c_a0ddf1224851353fc92bfbff6f499fa97}{main()}} method. You are free to modify your Cocoa app with generally no consequence to S\+DL. You cannot, however, easily change the S\+DL window itself. Functionality may be added in the future to help this.
\end{DoxyItemize}
\end{DoxyItemize}

Known bugs are listed in the file \char`\"{}\+B\+U\+G\+S.\+txt\char`\"{}. 